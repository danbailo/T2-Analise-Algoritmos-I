\documentclass[a4paper, 12pt]{article}
\usepackage[brazil]{babel}
\usepackage[utf8]{inputenc}
\usepackage{amsmath}
\usepackage{indentfirst}
\usepackage{graphicx}
\usepackage{caption}
\usepackage{subcaption}
\usepackage{multicol,lipsum}
\usepackage{cite}
\usepackage{placeins}
\usepackage{amsmath,amssymb,amsfonts}
\usepackage{algorithmic}
\usepackage{float}
\usepackage{textcomp}
\usepackage{xcolor}
\usepackage{enumerate}
\usepackage{wrapfig}
\usepackage{cite}
\usepackage{tikz}
\usepackage{color}   %May be necessary if you want to color links
\usepackage{hyperref}
\hypersetup{
    colorlinks=true,
    citecolor=blue,
    filecolor=blue,
    linkcolor=blue,
    urlcolor=blue
}
\hfuzz=66.002pt

\usepackage{listings}
\usepackage{color}

\begin{document}
\begin{titlepage}
    \begin{center}
		\LARGE{Universidade Federal de Mato Grosso do Sul}\\
		\vspace{15pt}
        \large{Campus Ponta Porã}\\ 
        \large{{\textbf{Análise de Algoritmos I}}}\\ 
        \vspace{15pt}
        \vspace{95pt}
        \textbf{\large{Trabalho Prático II}}\\
        \vspace{15pt}
        \textbf{\LARGE{Mochila Booleana}}\\
        %\title{{\large{Título}}}
        \vspace{3,5cm}
    \end{center}
    
    \begin{flushleft}
        \begin{tabbing}
            Aluno: Daniel de Leon Bailo da Silva\\            
            RGA: 2017.1805.021-6\\
            Professor: Eduardo Theodoro Bogue\\
            %Professor co-orientador: \\
    \end{tabbing}
 \end{flushleft}
    \vspace{1cm}
    
    \begin{center}
        \vspace{\fill}
            Maio\\
         2019
            \end{center}
\end{titlepage}

\newpage
\tableofcontents
\thispagestyle{empty}

\newpage
\pagenumbering{arabic}
\section{Resumo}
Este trabalho consiste em mostrar os resultados obtidos a partir da execução 
do algoritmo da {\it Mochilha Boolena} ou {\it Knapsack 0/1}, em suas versões dinâmicas.
Feito isso, dada as instâncias para realizar os testes, foi comparado o tempo de execução do algoritmo em questão, nas suas duas versões dinâmicas, {\it Top-Down} e  {\it Bottom-Up}.

Este trabalho possui um repositório online para melhor controle do versionamento e testes conforme a mudança do programa final.\\
\url{https://github.com/danbailo/T2-Analise-Algoritmos-I}
\newpage

\section{Knapsack Top-Down}


\begin{lstlisting}[basicstyle=\footnotesize, language=Python]
def topDown(number_items, weight_max, values_items, weight_items):
    if number_items == 0 or weight_max == 0: return 0
    if weight_items[number_items-1] > weight_max: 
     return topDown(number_items-1, weight_max, values_items, weight_items)
    if mem[number_items][weight_max] is not False: 
     return mem[number_items][weight_max]
    
    temp = max(topDown(number_items-1, 
    weight_max-weight_items[number_items-1],values_items, weight_items)
    +values_items[number_items-1], 
    topDown(number_items-1, weight_max, values_items, weight_items))
    
    mem[number_items][weight_max] = temp
    return temp    
\end{lstlisting}

\begin{figure}[h]
    \centering
    \begin{minipage}{.6\textwidth}
      \centering
      \includegraphics[width=1\linewidth]{../img/scatter_topDown.pdf}
      \captionof{figure}{A figure}
      \label{fig:test1}
    \end{minipage}%
    \begin{minipage}{.6\textwidth}
      \centering
      \includegraphics[width=1\linewidth]{../img/result_topDown.pdf}
      \captionof{figure}{Another figure}
      \label{fig:test2}
    \end{minipage}
    \end{figure}

\begin{figure}[h]
    \centering
    \includegraphics[width=1\textwidth]{../img/scatter_topDown.pdf}
    \caption{Gráfico do Tempo/Resultado}
    \label{fig:scatter_topDown}
\end{figure}

\newpage
\section{Knapsack Bottom-Up}
\begin{lstlisting}[basicstyle=\footnotesize, language=Python]
    def bottomUp(number_items, weight_max, values_items, weight_items): 
        K = [[0 for x in range(weight_max + 1)] for x in range(number_items + 1)]
        for i in range(number_items + 1): 
            for w in range(weight_max + 1): 
                if i == 0 or w == 0: K[i][w] = 0
                elif weight_items[i-1] <= w: K[i][w] = max(values_items[i-1] + 
                    K[i-1][w-weight_items[i-1]],  K[i-1][w]) 
                else: K[i][w] = K[i-1][w] 
        return K[number_items][weight_max]
    \end{lstlisting}
\newpage    

\section{Resultados Obtidos}
\end{document}